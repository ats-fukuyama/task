\documentclass[11pt]{article}
\usepackage{af}
\usepackage{sym}
\usepackage{doc}

\begin{document}
\begin{flushright}
2019-09-28
\end{flushright}

\begin{center}
{\Large \bf Parallel processing and matrix solver interface: mtxp}
\end{center}

\tableofcontents


\section{Parallel processing library: MPICH}

\subsection{Download}

\begin{enumerate}
\item
Download a source file from MPICH web: 
\begin{quote}
\begin{verbatim}
https://www.mpich.org/downloads/
mpich-3.3.1.tar.gz (26MB)
\end{verbatim}
\end{quote}
\item
Make a directory and expand the source file
\begin{quote}
\begin{verbatim}
mkdir mpich
cd mpich
mv ~/Download./mpich-3.3.1.tar.gz .
tar xzf mpich-3.3.1.tar.gz
cd mpich-3.3.1
\end{verbatim}
\end{quote}
\end{enumerate}

\subsection{Configure}
\begin{enumerate}
\item
Create a configure script ``run'' by your favarite editor

\begin{quote}
\begin{verbatim}
CC=gcc CFLAGS="-m64" CXX=c++ CXXFLAGS="-m64" FC=gfortran
FFLAGS="-m64" ./configure --prefix=/usr/local/mpich331-gfortran-gcc
--enable-cxx --enable-fast --enable-romio --disable-shared
\end{verbatim}
Install directory: \verb|/usr/local/mpich331-gfortran-gcc|
\end{quote}

\item
Make ``run'' executable 
\begin{quote}
\begin{verbatim}
chmod 755 run
\end{verbatim}
\end{quote}

\item
Configure
\begin{quote}
\begin{verbatim}
./run
\end{verbatim}
\end{quote}
\end{enumerate}

\subsection{Compile and install}
\begin{enumerate}
\item
Compile
\begin{quote}
\begin{verbatim}
make
\end{verbatim}
\end{quote}
\item
Install
\begin{quote}
\begin{verbatim}
sudo make install
\end{verbatim}
\end{quote}
\end{enumerate}


\section{Parallel matrix solver library: PETSc}

\subsection{Download}
\begin{enumerate}
\item
Download a source file from PETSc web site
\begin{quote}
\begin{verbatim}
https://www.mcs.anl.gov/petsc/download/
petsc-3.11.3.tar.gz (33.2MB)
\end{verbatim}
\end{quote}
\item
Make a directory and expand the source file
\begin{quote}
\begin{verbatim}
sudo mkdir /opt
cd /opt
sudo mkdir petsc
sudo chown admin:admin petsc \qquad ("admin" should be replaced by you
user name)
cd petsc
mv ~/Download/petsc-3.11.3.tar.gz .
tar xzf petsc-3.11.3.tar.gz
\end{verbatim}
\end{quote}
\end{enumerate}

\subsection{Configure}
\begin{enumerate}
\item
Create a configure script ``gfortran.py'' by your favarite editor
\begin{quote}
\hrule
\begin{verbatim}
#!/usr/bin/env python

# Build PETSc, with gfortran

configure_options = [
  '--with-mpi=1',
  '--with-mpi-dir=/usr/local/mpich331-gfortran-gcc',
  '--with-shared-libraries=0',
  '--with-cxx-dialect=C++11',
  '--download-mpich=0',
  '--download-hypre=0',
  '--download-fblaslapack=1',
  '--download-spooles=1',
  '--download-superlu=1',
  '--download-metis=1',
  '--download-parmetis=1',
  '--download-superlu_dist=1',
  '--download-blacs=1',
  '--download-scalapack=1',
  '--download-mumps=1'
  ]

if __name__ == '__main__':
  import sys,os
  sys.path.insert(0,os.path.abspath('config'))
  import configure
  configure.petsc_configure(configure_options)
\end{verbatim}
\hrule
\end{quote}
\item
Make ``gfortran.py'' executable
\begin{quote}
\begin{verbatim}
chmod 755 gfortran.py
\end{verbatim}
\end{quote}
\item
Setup environment variables for PETSc
\begin{quote}
\begin{verbatim}
export PETSC_DIR=/opt/petsc/petsc-3.11.3
export PETSC_ARCH=gfortran
\end{verbatim}
\end{quote}
\item
Configure
\begin{quote}
\begin{verbatim}
./gfortran.py
\end{verbatim}
\end{quote}
\end{enumerate}

\subsection{Compile}
\begin{enumerate}
\item
Compile
\begin{quote}
\begin{verbatim}
make all
\end{verbatim}
\end{quote}
\end{enumerate}

\section{Compile of task/mtxp module}

\subsection{Update setup file}
\begin{enumerate}
\item
Goto mtxp directory
\begin{quote}
\begin{verbatim}
cd task/mtxp
\end{verbatim}
\end{quote}
\item
Create a setup file 
\begin{quote}
\begin{verbatim}
cp make.mtxp.org make.mtxp
\end{verbatim}
\end{quote}
\item
Edit the setup file ``make.mtxp''
\begin{itemize}
\item
If MPI is not available, 
remove comment mark ``\#'' on lines 4--9
\item
If MPI is available but PETs not, 
remove comment mark ``\#'' on lines 12--17
\item
If MPI and PETSc are available, 
remove comment mark ``\#'' on lines 21, 38,39, 41--45
\end{itemize}
\end{enumerate}
\subsection{Compile}
\begin{enumerate}
\item
Compile
\begin{quote}
\begin{verbatim}
make
\end{verbatim}
\end{quote}
\end{enumerate}

\section{Compile of task/fp and related modules}

\subsection{Update Makefile}
\begin{enumerate}
\item
Change directory
\begin{quote}
\begin{verbatim}
cd ../fp
\end{verbatim}
\end{quote}
\item
Edit Makefile
\begin{itemize}
\item
To use serial band matrix solver, remove comment mark ``\#'' on lines 4-5.
\item
To use serial iterative solver, remove comment mark ``\#'' on lines 6-7.
\item
To use parallel direct solver MUMPS, remove comment mark ``\#'' on lines 8-9.
\item
To use parallel direct solver library PETSc, remove comment mark ``\#''
on lines 10-11. .
\end{itemize}
\end{enumerate}

\subsection{Compile}
\begin{enumerate}
\item
Compile related modules and task/fp files
\begin{quote}
\begin{verbatim}
make
\end{verbatim}
\end{quote}
\end{enumerate}
\end{document}

make

