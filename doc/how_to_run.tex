\documentclass[12pt]{article}
\usepackage{geometry}
\geometry{a4paper}
\usepackage[dvips]{graphicx,color}
\usepackage{amsmath}
\usepackage{amssymb}
\usepackage[varg]{txfonts}
\usepackage{exscale}
\usepackage{af}
\usepackage{slide}
\usepackage{card}
\usepackage{sym}
\usepackage{ulem}

\newcommand{\epspath}{/Users/fukuyama/tex/eps}
\newcommand{\ttype}[1]{\emC{\LARGE \texttt{#1}}}
\newcommand{\tline}[1]{\qquad{\Large \texttt{#1}}\\[-3mm]}

%\let\uetheta=\ueth
%\let\uephi=\ueph
\let\Ra=\Rightarrow
\let\Lra=\Longrightarrow

\newenvironment{flist}{\par\noindent\begin{tabular}{%
p{45mm}p{35mm}p{30mm}p{120mm}}}{\end{tabular}\par\bigskip}
\newcommand{\fitem}[4]{\quad #1 & #2 & #3 & #4 \\}
\newcommand{\ffitem}[2]{\multicolumn{3}{p{109mm}}{#1} & #2 \\}
\newcommand{\fffitem}[1]{\multicolumn{4}{p{228mm}}{#1} \\}

\newcommand{\vvd}{\vv_\mr{d}}
\newcommand{\vd}{v_\mr{d}}
\newcommand{\vvE}{\vv_E}
\newcommand{\Epara}{E_\para}
\newcommand{\Tperp}{T_\perp}
\newcommand{\Tb}{T_\mr{b}}
\newcommand{\kparam}{k_{\para m}}
\newcommand{\energy}{\varepsilon}
\newcommand{\tx}{TASK/TX}

\newcommand{\xtitle}[2]{%
{\fboxsep=10mm \fboxrule=3mm
\cfbox{dbback}{#1}{%
\begin{center}\Huge
 \emC{#2}
\end{center}
}}
}

\begin{document}
\huge\sf

\begin{flushleft}
\fig{0.70}{\epspath/logo/KUlogo.eps}
\end{flushleft}
\vspace*{-40mm}

\begin{flushright}\LARGE
  2022/04/11
\end{flushright}
\vspace{20mm}

\xtitle{160mm}{%
Introduction to running TASK
}
\medskip

\begin{center}\huge
\emB{A.~Fukuyama}
\\[8mm]
\textcolor{fgreen}{\huge
Professor Emeritus, Kyoto University 
}
\end{center}
\medskip

\begin{center}
  \begin{minipage}{180mm}
    \begin{enumerate}
    \item
      Basic operation
    \item
      How to setting parameters
    \item
      Operation of equilibrium module: TASK/EQ
    \item
      Operation of transport module: TASK/TR
    \item
      Coupled operation of TASK/EQ and TASK/TR
    \end{enumerate}
  \end{minipage}
\end{center}

\title{Basic operation}
\begin{itemize}
\item
  \emA{Setup of graphic library GSAF}
  \begin{itemize}
  \item
    \emB{At the beginning of TASK codes, setup of GSAF is required.}
    \begin{itemize}
    \item
      \emC{Select graphic resolution} (0: metafile only, no graphics, n:)
    \item
      \emC{Input one character command}
      \begin{itemize}
      \item
        c: continue
      \item
        f: set metafile name (e.g. xxx.gs) and start saving
      \end{itemize}
    \end{itemize}
  \end{itemize}
\item
  \emA{Choice of graphic operation}
  \begin{itemize}
  \item
    \emB{At the end of one page drawing, choose one char. command}
    \begin{itemize}
    \item
      \emC{c} or \emC{CR}: change focus to original window and continue
    \item
      \emC{f}: set metafile name and start saving
    \item
      \emC{s}: start saving and save this page
    \item
      \emC{y}: save this page and continue
    \item
      \emC{n}: continue without saving
    \item
      \emC{d}: dump this page as a bitmap file ``gsdump\textit{n}''
%    \item
%      \emC{b}: switch on/off bell sound
    \item
      \emC{q}: quit program after confirmation
    \end{itemize}
  \end{itemize}
\end{itemize}
  

\title{Graphic Utilities}
\begin{itemize}
\item
\emA{Utility program}
\begin{itemize}
\item
\emB{gsview}: View metafile
\item
\emB{gsprint}: Print metafile on a postscript printer
\item
\emB{gstoeps}: Convert metafile to eps files of each page
\item
\emB{gstops}: Convert metafile to a postscript file of all pages
%\item
%\emC{gstotgif}: Convert metafile to a tgif file for graphic editor tgif
%\item
%\emC{gstotsvg}: Convert metafile to a svg file for web browser
\end{itemize}
\item
\emA{Options}
\begin{itemize}
\item
\emC{-a}: output all pages, otherwise interactive mode
\item
\emC{-s ps}: output from page ps
\item
\emC{-e pe}: output until page pe
\item
\emC{-p \textit{np}}: output contiguous \textit{np} pages on a sheet
\item
\emC{-b}: output without title
%\item
%\emC{-r}: rotate page
%\item
%\emC{-z}: gray output
\end{itemize}
\item
  \emA{Example}
  \begin{itemize}
    \item
      \emC{gstops -ab xxx.gs}: convert all figures to one postscript file
    \item
      \emC{gstoeps -ab xxx.gs}: convert each figure to a eps file
  \end{itemize}
\end{itemize}

\title{Typical File Name of TASK}
\begin{itemize}
\itemsep=3mm
\item
\ttype{XXcomm.f90}: Definition of global variables, allocation of arrays
\item
\ttype{XXmain.f90}: Main program for standalone use, read XXparm file
\item
\ttype{XXmenu.f90}: Command input
\item
\ttype{XXinit.f90}: Default values
\item
\ttype{XXparm.f90}: Read input parameters
\item
\ttype{XXview.f90}: Show input parameters
\item
\ttype{XXprep.f90}: Initialization of run, initial profile
\item
\ttype{XXexec.f90}: Execution of run
\item
\ttype{XXgout.f90}: Graphic output
\item
\ttype{XXfout.f90}: Text file output
\item
\ttype{XXsave.f90}: Binary file output
\item
\ttype{XXload.f90}: Binary file input
\end{itemize}

\title{Typical input command}
\begin{itemize}
\item
  When input line includes \emA{=}, interpreted as a namelist
  input (e.g., \ttype{rr=6.5})
\item
  When the first character is not an alphabet, interpreted as a
  one-character command
  \begin{itemize}
  \item
    \ttype{r}: Initialize profiles and execute
  \item
    \ttype{c}: Continue run
  \item
    \ttype{p}: Namelist input of input parameters
  \item
    \ttype{v}: Display of input parameters
  \item
    \ttype{g}: Graphic output
  \item
    \ttype{w}: print output
  \item
    \ttype{s}: Save results into a file
  \item
    \ttype{l}: Load results from a file
  \item
    \ttype{q}: End of the program
  \end{itemize}
\end{itemize}

\title{How to setting input parameters}
\begin{itemize}
\item
  \emA{Default setting of the module}
  \begin{itemize}
  \item
    Default parameters are set at the subroutine \ttype{XX\_init} in
    \ttype{XXinit.f90}
  \end{itemize}
\item
  \emA{Preset parameter file}
  \begin{itemize}
  \item
    If there is a namelist file \ttype{XXparm} in the executing
    directory, the module reads the file after the default parameter
    setting in \ttype{XX\_init}.
  \item
    Be careful not to leave an unnecessary file \ttype{XXparm} for
    avoiding unintentional set of parameters.
  \end{itemize}
\item
  \emA{Setup of parameters by input lines}
  \begin{itemize}
  \item
    You can set parameters by an input line of the namelist form \\
    \quad  \ttype{name1=value1, name2=value2, name3=value3}
  \item
    You can set a list of parameters by the input lines after the one
    char. command ``p'' in the form of namelist file \\
    \quad  \ttype{\&XX} \\
    \quad  \ttype{name1=value1, name2=value2, name3=value3}
    \quad \emBs{(more lines)}\\
    \quad  \ttype{/}
  \end{itemize}
\end{itemize}

\title{How to run TASK/EQ (1)}

\begin{itemize}
\item
  \emA{Interactive operation with default parameters}
  \begin{itemize}
  \item
    \quad \ttype{cd task/eq}
  \item
    \emBs{If there is a file named \ttype{eqparm} in this directory,
      remove or rename it}
  \item
    \emBs{On macOS, start the module from a} \emAs{xterm}
    \emBs{window, not from a terminal window.}
  \item
    Key input sequence \\
    \quad \ttype{./eq} \qquad \emBs{start eq module)}\\
    \quad \ttype{5} \qquad \emBs{(window size 1024x760)}\\
    \quad \ttype{c} \qquad \emBs{(continue operation)}\\
    \quad \ttype{r} \qquad \emBs{(run the module with default setting)}\\
    \quad \ttype{g} \qquad \emBs{(start eq graphic interface)}\\
    \quad \ttype{s2}\qquad \emBs{(2D standard plot)}\\
    \quad \ttype{CR,CR,CR,CR}\qquad \emBs{(repeat Carriage Return 4 times)}\\
    \quad \ttype{x} \qquad \emBs{(exit eq graphic interface)}\\
    \quad \ttype{q} \qquad \emBs{(quit eq module)}
  \end{itemize}
\end{itemize}

\title{How to run TASK/EQ (2)}

\begin{itemize}
\item
  \emA{Interactive operation with a preset file}
  \begin{itemize}
  \item
    \quad \ttype{cd task/eq}
  \item
    \quad \ttype{cp parm/eqparm.ITER eqparm}
  \item
    \emBs{On macOS, start the module from a} \emAs{xterm}
    \emBs{window, not from a terminal window.}
  \item
    Key input sequence \\
    \quad \ttype{./eq} \qquad \emBs{start eq module)}\\
    \quad \ttype{5} \qquad \emBs{(window size 1024x760)}\\
    \quad \ttype{c} \qquad \emBs{(continue operation)}\\
    \quad \ttype{r} \qquad \emBs{(run the module with default setting)}\\
    \quad \ttype{g} \qquad \emBs{(start eq graphic interface)}\\
    \quad \ttype{s} \qquad \emBs{(1d and 2D standard plot)}\\
    \quad \ttype{CR}$\times 10$ \qquad \emBs{(repeat Carriage Return 10 times)}\\
    \quad \ttype{x} \qquad \emBs{(exit eq graphic interface)}\\
    \quad \ttype{q} \qquad \emBs{(quit eq module)}
  \end{itemize}
\item
  \quad \ttype{rm eqparm} \qquad \emBs{(remove eqparm file)}
\end{itemize}

\title{How to run TASK/EQ (3)}

\begin{itemize}
\item
  \emA{Batch operation with an input file}
  \begin{itemize}
  \item
    \quad \ttype{cd task/eq}
  \item
    Start the module which reads an existing input file
    (\ttype{in/eq.ITER01.in}), 
    writes an output file (\ttype{eq.ITER01.out}),
    and generates a graphic metafile (\ttype{eq.ITER01.gs}) \\
    \quad \ttype{./eq <in/eq.ITER01.in >eq.ITER01.out}
  \item
    View the graphic output \\
    \emBs{On macOS, start from a} \emAs{xterm}
    \emBs{window, not from a terminal window.} \\
    \quad \ttype{gsview eq.ITER01.gs} \\
    Enter figure page number or 0 for all
  \item
    Convert the graphic metafile to a postscript file \\
    \quad \ttype{gstops -ab eq.ITER01.gs >eq.ITER01.ps}
  \item
    Convert all the figures in the graphic metafile to EPS files \\
    \quad \ttype{gstoeps -ab eq.ITER01.gs}
  \end{itemize}
\end{itemize}

\title{How to run TASK/TR (1)}

\begin{itemize}
\item
  \emA{Interactive operation with default parameters}
  \begin{itemize}
  \item
    Example of key input sequence \\
    \quad \ttype{cd task/tr} \\
    \quad \ttype{./tr2} \qquad \emBs{(start tr module: tr may conflict
      with a unix command)}\\
    \quad \ttype{5} \qquad \emBs{(window size 1024x760)}\\
    \quad \ttype{c} \qquad \emBs{(continue operation)}\\
    \quad \ttype{r} \qquad \emBs{(start a run with default setting)}\\
    \quad \ttype{g} \qquad \emBs{(start tr graphic interface)}\\
    \quad \ttype{t6} \qquad \emBs{(time evolution) CR for next graphic input}\\
    \quad \ttype{r1} \qquad \emBs{(radial profile)}\\
    \quad \ttype{x} \qquad \emBs{(exit tr graphic interface)} \\
    \quad \ttype{c} \qquad \emBs{(continue the run)} \\
    \quad \ttype{g} \qquad \emBs{(start tr graphic interface)}\\
    \quad \ttype{t6} \qquad \emBs{(time evolution)}\\
    \quad \ttype{r1} \qquad \emBs{(radial profile)}\\
    \quad \ttype{x} \qquad \emBs{(exit tr graphic interface)} \\
    \quad \ttype{q} \qquad \emBs{(quit eq module)}
  \end{itemize}
\end{itemize}

\title{How to run TASK/TR (2)}

\begin{itemize}
\item
  \emA{Interactive operation with a preset file}
  \begin{itemize}
  \item
    \quad \ttype{cd task/eq}
  \item
    \quad \ttype{cp parm/trparm.ITER trparm}
  \item
    \quad \ttype{./tr2} \qquad \emBs{start tr module)}\\
    \quad \ttype{5} \qquad \emBs{(window size 1024x760)}\\
    \quad \ttype{c} \qquad \emBs{(continue operation)}\\
    \quad \ttype{r} \qquad \emBs{(run the module with default setting)}\\
    \quad \ttype{ntmax=1000 nbtot=25} \qquad \emBs{(change parameters)}\\
    \quad \ttype{c} \qquad \emBs{(continue run)}\\
    \quad \ttype{g} \qquad \emBs{(start eq graphic interface at 6 s)}\\
    \quad \ttype{t6} \qquad \emBs{(time evolution, CR for next input)}\\
    \quad \ttype{r1} \qquad \emBs{(radial profile)}\\
    \quad \ttype{x} \qquad \emBs{(exit eq graphic interface)}\\
    \quad \ttype{q} \qquad \emBs{(quit eq module)}
  \item
    \quad \ttype{rm trparm} \qquad \emBs{(remove trparm file)}
  \end{itemize}
\end{itemize}

\title{How to run TASK/EQ (3)}

\begin{itemize}
\item
  \emA{Batch operation with an input file}
  \begin{itemize}
  \item
    Input file: \ttype{in/tr.ITER01.in}
  \item
    This example uses a fixed-boundary equilibrium file
    \ttype{../eq/eqdata.ITER01} in a previous eq run
  \item
    \quad \ttype{cd task/eq}
  \item
    \quad \ttype{./tr2 <in/tr.ITER01.in | tee tr.ITER01.out}
  \item
    \quad \ttype{gsview tr.ITER01.gs}
  \end{itemize}
\end{itemize}

\end{document}
