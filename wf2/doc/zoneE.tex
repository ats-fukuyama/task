\section{Creation of attribute data ({\tt wfzone})}
\begin{enumerate}
\item
Function
\begin{itemize}
\item[Z:]
creation of attribute data
\item[G:]
graphic output of attribute data
\item[W:]
numerical output of attribute data
\item[S:]
save attribute data to a file
\item[L:]
load attribute data from a file
\item[X:]
exit
\end{itemize}

\item
Creation of attribute data
\begin{itemize}
\item[D:]
Definition of element attribute data
\begin{itemize}
\item
NDMAX: number of data
\item
ND,ID,XMIN,XMAX,YMIN,YMAX: element zone number, attribute of element,
zone area
\item
Define element attribute of the elements in the zone
\begin{itemize}
\item
ID=0: plasma (default)
\item
ID=1: vacuum
\item
ID$>$1: dielectric of specified dielectric attribute
\end{itemize}
\end{itemize}

\item[E:]
Definition of dielectric attribute
\begin{itemize}
\item
NMMAX: number of dielectric attribute
\item
NM,EPSD: dielectric attribute number, relative dielectric constant
\item
Define relative dielectric constant for dielectric attribute number
\end{itemize}

\item[M:]
Definition of permitive/dielectric/resistive attribute
\begin{itemize}
\item
NMMAX: number of material attribute
\item
NM: dielectric attribute number
\item
EPSD: relative dielectric constant
\item
RMUD: relative permittivity constant
\item
SIGD: resisivity
\end{itemize}

\item[A:]
Definition of wave boundary attribute
\begin{itemize}
\item
NBMAX: number of wave boundary attributes
\item
NB,ID,XMIN,XMAX,YMIN,YMAX: wave boundary attribute number, type of
wave boundary attribute, zone area
\item
Define wave boundary attribute of the boundary nodes in the zone
\begin{itemize}
\item
ID=0: in the plasma (default)
\item
ID=1: boundary between plasma and conductor with potential 0
\item
ID=2: boundary between plasma and conductor with potential PHIW
\item
ID=3: boundary between plasma and conductor with potential continuously
varying from 0 to PHIW
\item
\end{itemize}
\end{itemize}

\item[B:]
Define transport boundary attribute
\begin{itemize}
\item
NBMAX: number of transport boundary attributes
\item
NB,ID,XMIN,XMAX,YMIN,YMAX: transport boundary attribute number,
transport boundary attribute, zone area
\item
Define transport boundary attribute of the boundary nodes in the zone
\begin{itemize}
\item
ID=0: in a plasma (default)
\item
ID=1: boundary between plasma and conductor with potential 0
\item
ID=2: boundary between plasma and conductor with potential PHIES
\item
ID=3: boundary between plasma and conductor with potential
continuously varying from 0 to PHIES.
\item
ID=4: boundary between plasma and conductor with potential oscillating
with amplitude PHIES and frequency RFES
\item
ID=5: boundary between plasma and conductor with potential oscillating
with amplitude varying form 0 to PHIES and frequency RFES
\item
ID=8: boundary between plasma and dielectric (insulator) without
surface charge
\item
ID=9: boundary between plasma and dielectric (insulator) with
surface charge
\item
ID=10: vacuum or dielectric (default)
\item
ID=11: boundary between vacuum/dielectric and conductor with potential 0
\item
ID=12: boundary between vacuum/dielectric and conductor with potential PHIES
\item
ID=13: boundary between vacuum/dielectric and conductor with potential
continuously varying from 0 to PHIES.
\item
ID=14: boundary between vacuum/dielectric and conductor with potential oscillating
with amplitude PHIES and frequency RFES
\item
ID=15: boundary between vacuum/dielectric and conductor with potential oscillating
with amplitude varying form 0 to PHIES and frequency RFES
\end{itemize}
\end{itemize}

\item[W:]
Definition of wave boundary potential variation
\begin{itemize}
\item
NVMAX: number of wave boundary potential variation
\item
NV,XMIN,XMAX,YMIN,YMAX: wave boundary potential variation number, wave
boundary potential variation, zone area
\item
For wave boundary attribute ID=3, define the range of
wave potential varying
\\\qquad only NV=1 is available at present
\\\qquad wave potential = 0 at X=XMIN, Y=YMIN
\\\qquad and PHIW at X=XMAX, Y=YMAX
\end{itemize}

\item[P:]
Definition of transport boundary potential variation
\begin{itemize}
\item
NVMAX: number of transport boundary potential variation
\item
NV,XMIN,XMAX,YMIN,YMAX: transport boundary potential variation number,
transport boundary potential variation, zone area
\item
For wave boundary attribute ID=3, 5, 13, 15, define the range of
transport potential varying
\\\qquad only NV=1 is available at present
\\\qquad wave potential = 0 at X=XMIN, Y=YMIN
\\\qquad and PHIS at X=XMAX, Y=YMAX
\end{itemize}

\item[V:]
view attributes
\item[X:]
exit
\end{itemize}
\end{enumerate}

