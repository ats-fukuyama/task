\documentclass[11pt]{article}
\usepackage{../../doc/af}
\usepackage{../../doc/sym}
\usepackage{../../doc/doc}

\begin{document}
\begin{flushright}
2023/02/06
\end{flushright}

\begin{center}
\textbf{\Large Computing scheme of poloidal-coil current in task/equ}
\end{center}

\begin{itemize}
\item
  Variables:
  \begin{itemize}
  \item
    Marker point position: $\vr_j^s = (R_j^s, Z_j^s)$; $j=1..J^s$
  \item
    Poloidal coil current (initial value): $I_k^{v0}$; $k=1..K^v$
  \item
    Poloidal magnetic flux generated by unit coil current: $\psi_k^v$;
    $k=1..K^v$
  \item
    Weight of marker points: $w_j^s$; $j=1..J^s$
  \item
    Weight of poloidal coil current: $w_k^v$: $k=1..K^v$
  \item
    Poloidal magnetic flux generated by the plasma current: $\psi^p$
  \item
    Poloidal magnetic flux prescribed at the marker points: $w_j^s$; $j=1..J^s$
  \end{itemize}
   
\item
  Condition to be satisfied:
  \begin{itemize}
  \item
    Poloidal current $I_k^v$ ($k=1,K^v$) is determined by minimizing
    the variation
    \[
    \delta W(I_k^v) = \sum_{j=1}^{J^s} w_j^s
    \left(\psi^p(\vr_j^s)+\sum_{k=0}^{K_v} I_k^v \psi^k(\vr_j^s)
    -\psi_j^m\right)^2
    +\sum_{k=1}^{K^v} w_k^v(I_k^v-I_k^{v0})^2
    \]
  \end{itemize}

\item
  Additional constraints:
  \begin{itemize}
    \itemsep=5pt
  \item
    Usually the marker points are set on the plasma surface, and the
    poloidal magnetic flux at the marker points are given as
    $\psi_j^m=0$.  For this purpose, we set $\psi_0^v=1$ and an
    unknown constant $I_0^v$ is introduced for $\psi$ to be $0$ on the
    plasma surface.
  \item
    If we define the poloidal magnetic flux $\psi=0$ at the torus
    center $R=0$, $I_0^v$ corresponds to the poloidal magnetic flux on
    the plasma surface.
  \item
    When a poloidal coil current is fixed to a constant externally, it
    is better to include the poloidal magnetic flux generated by this
    coil current into the poloidal magnetic flux generated by the
    plasma current, and exclude it from the variation target, rather
    than increasing the weight $w_k^v$ of the poloidal magnetic flux.
  \item
    When the analysis is based on the magnetic measurement data,
    the marker points can be set at the measurement position, and the
    measured magnetic flux is set as the poloidal magnetic flux at the
    marker points. Then the analysis becomes a completely
    free-boundary problem.
  \end{itemize}

\item
  Additional comments for the analyses coupled with transport
  \begin{itemize}
    \itemsep=5pt
  \item
    When the plasma position is controlled, the marker points should
    be external input parameters.
  \item
    When necessary, such as the plasma current ramp-up phase, the
    poloidal coil current usually fixed may be pre-program controlled.
  \item
    In the present analysis, the toroidal magnetic flux on the plasma
    surface is assumed to be constant.  Therefore the plasma surface
    may be expanded or compressed in time so that the surface current
    due to the toroidal magnetic field is canceled.
  \end{itemize}
\end{itemize}

\end{document}

